%! tex-root=main.tex
\section{Интеграл Римана}

\subsection{Найти интегралы и нарисовать площади (формула. Ньютона-Лейбница)}
\textbf{3.1.4} \\
\begin{align*}
  \int_{\frac{1}{\sqrt{3}}}^{\sqrt{3}} \frac{dx}{1+2^2} = 
    \arctg{x}\Big|^{\sqrt{3}}_{\frac{1}{\sqrt{3}}} = 
    \arctg{\sqrt{3}} - \arctg{\frac{1}{\sqrt{3}}} = \frac{\pi}{3} - \frac{\pi}{6} =
    \frac{\pi}{6}
.\end{align*}
\subsection{Объяснить, почему формальное применение ф. Ньютона-Лейбница приводит к неверным результатам}
\subsection{С помощью определенных интегралов посчитать пределы следующих сумм}
\textbf{3.3.3} \\
\begin{align*}
\lim_{n \to \infty} \left( \sin \frac{\pi}{n} + \sin \frac{2\pi}{n} + \ldots + \sin \frac{(n-1)\pi }{n} \right) 
&= \lim_{n \to \infty} \int_{1}^{n-1} \sin \frac{\pi u}{n} du = \\
&= \lim_{n \to \infty} \left( \sin \frac{\pi u}{n}\Big|^{n-1}_1 \right) = \\
&= \lim_{n \to \infty} \left( \sin \pi - \sin \frac{-\pi }{n} - \sin\frac{\pi}{n} \right) =
\boxed{\sin \pi}
.\end{align*}
\subsection{Вычислить интегралы}
\textbf{3.4.4} \\
\columnratio{0.7}
\begin{paracol}{2}
\begin{align*}
  &\int_{0}^{2\pi} \frac{dx}{\sin ^4x+\cos ^4x} 
  = 4 \int_{0}^{2\pi} \frac{dx}{3+\cos 4x} \xrightarrow[4x \to x]{} 
  4 \int_{0}^{2\pi} \frac{dx}{3+\cos x} = \\
  &= 8 \int_{0}^{\pi} \frac{dx}{3+\cos x} = 
16 \int_{0}^{\infty} \frac{1}{3+\frac{1-t^2}{1+t^2}\frac{dt}{1+t^2}} = \\
  &= 8 \int_{0}^{\infty} \frac{dt}{t^2+2} =
\frac{8}{\sqrt{2}}\arctg{\frac{t}{\sqrt{2}}}\Big|^{\infty}_{0} = \frac{4\pi}{\sqrt{2}}
.\end{align*}
\switchcolumn%
\begin{align*}
  \sin^2x&=\frac{1-\cos2x}{2} \\
  \cos^2x&=\frac{1+\cos2x}{2} \\
  \sin^4x + \cos^4x &= \frac{1}{4}\cdot (3+\cos 4x) \\
  \tg{\frac{x}{2}} &= t \\
.\end{align*}
\end{paracol}

\subsection{Определить знак интеграла}
\subsection{Какой интеграл больше}
\subsection{Определить среднее значение ф-ий в промежутках}
\textbf{3.7.1} \\
\begin{align*}
  f(x) = x^2 на [0, 1]\\
  A(x) = \frac{1}{b-a}\int_{a}^{b} f(x)\dx = \int_{0}^{1} x^2 \dx = \frac{x^3}{3}\Big|^1_0 = \frac{1}{3}
.\end{align*}
\subsection{Вычислить}
\textbf{3.8.3} \\
\begin{align*}
  &\int_{-\infty}^{\infty} \frac{\dx}{1+x^2}
  = \lim_{r \to -\infty} \int_{r}^{0} \frac{dx}{1+x^2} + \lim_{R \to \infty}\int_{0}^{R}\frac{dx}{1+x^2} 
= \arctg{x}\Big|^0_r - \arctg{x}\Big|^R_0 = \frac{\pi}{2} + \frac{\pi}{2} = \pi 
.\end{align*}
\subsection{Исследовать на сходимость интегралы}
\textbf{3.9.2} \\
TODO
\begin{align*}
  \int_{1}^{\infty} \frac{dx}{x \sqrt[3]{x^2+1}} = \ldots 
.\end{align*}
\subsection{Найти площади фигур}
\begin{enumerate}
  \item $y = x^2, x + y = 2$
    \begin{align*}
      S = \int_{-2}^{1} (2-x)\dx - \int_{-2}^{1} x^2\dx = (2x-\frac{x^2}{2})\Big|^1_{-2} - \frac{x^3}{3}\Big|^1_{-2} =
      \frac{3}{2} + 6 - (\frac{1}{3} + \frac{8}{3}) = \frac{9}{2}
    .\end{align*}
  \item $y = 2x - x^2, x + y = 0$
    \begin{align*}
      S = \int_{0}^{3}(3x-x^2) \dx - \int_{0}^{3} (-x)\dx = \int_{0}^{3} (3x-x^2) \dx = 
      (\frac{3x^2}{2} - \frac{x^3}{3})\Big|^3_0 = \frac{27}{2} - 9 = \frac{9}{2}
    .\end{align*}
  \item $y=2^x, y = 2, x = 0$
    \begin{align*}
      S = \int_{0}^{1} 2\dx - \int_{0}^{1} 2^x\dx = 2 - \frac{2^x}{\ln 2}\Big|^1_0 = 2 - \frac{2}{\ln 2} + \frac{1}{\ln 2} = 2\cdot \frac{\ln 2-1}{\ln 2}  
    .\end{align*}
\end{enumerate}
